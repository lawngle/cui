\documentclass{article}
\usepackage{graphicx}

\title{Bimetallic Lanthanide and Actinide Metallocene Hydrides}
\author{Eric Ma, Thinh Nguyen, Manny Teutla}
\date{Chem 250L Winter 2024}

\begin{document}

\maketitle

\section{Abstract} % Last

\section{Introduction} % Third
The valence electrons on lanthanide and actinide cations were thought to not
participate in bonding due to the limited radial extent of the $f$-orbitals.
However, recent studies in Ln$^{2+}$ complexes revealed that lignad field
splitting can lower the energy of $d$-orbitals and lead to the formation of
bimetallic lanthanide and actinide complexes.  The aim of this experiment is to
investigate the possibility of Y-Y bonding in the recently synthesized
bimetallic complex [(Cp)$_2$Y($\mu - $H)]$_2$. \\
% Introduce the works used related to this experiment.

\section{Methods} % Start Here
\subsection{Statement of the Models} % Need to check template
% Insert model of 4f and 5d radial distribution functions
In a lanthanide, the 4$f$ orbitals are much closer to the nuclei compared to the
5d orbital. % Need to fact check this.
Electrons removed fom the system will come from the 5$d$ atomic orbital first,
increasing the energy of the orbital while not affecting the 4$f$ atomic orbital
as much.  As the oxidation state increaes, the energy separation between the two
orbitals decreases. \\ Orbital localization is expected to have no effect on the
total energy of a molecule, nor the molecular structure.  The orbital energies
are expected to increase when localization is applied.
% Expectation of orbital localization being applied to a hypovalent compound

\subsection{Computational Details}
% Please include any non-default parameters you guys set :3
A DFT calculation was performed on B$_2$H$_6$ using the TPSSh functional and
def2-SVP basis sets.  The diborane molecule was assigned D$_{2h}$ symmetry.
Boys localization was performed after lowering to $C_1$ symmetry.  The canonical
Kohn-Sham molecular orbitals and Boys localized orbitals were plotted.

Ground state optimizations of [(Cp)$_2$Y($\mu - $H)]$_2$ and its monoanion were
performed at the TPSSh level using def2-SVP basis sets.  Solvation of the
complexes in THF was simulated with a dielectric constant of $7.58$.  The
HOMO-LUMO gaps of different spin states were analyzed to determine which one the
molecule would be in.  The spin states of the complexes was used to determine
whether the complex would be EPR active.  Frontier orbitals were visualized to
deterime whether metal-metal bonding was observe
% Compare diborane and lanthanide units later.

\section{Results and Discussion} % Second

\begin{table}[htbp]
    \centering
    \caption{}
    \begin{tabular}{c|ccc|ccc}
	    & \multicolumn{3}{c}{Canonical} & \multicolumn{3}{c}{Localized} \\
	Orbital & Assignment & Energy (eV) & Visualization & Assignment & Energy (eV) & Visualization \\
	    \hline
      HOMO (1b2g)   &  $\pi^*_{\mathrm{p-p}}$     &   -8.514      &
	    &  $\sigma^*_{\mathrm{s-p}}$  &  -10.726  \\
      HOMO-1 (3ag)  &  $\sigma_{\mathrm{p-p}}$    &   -9.867      &
	    &  $\sigma^*_{\mathrm{s-p}}$  &  -10.726  \\
      HOMO-2 (1b3u) &  $\pi_{\mathrm{p-p}}$       &  -10.156      &
	    &  $\sigma^*_{\mathrm{s-p}}$  &  -10.726  \\
      HOMO-3 (1b2u) &  $\pi_{\mathrm{p-p}}$       &  -11.516      &
	    &  $\sigma^*_{\mathrm{s-p}}$  &  -10.726  \\
      HOMO-4 (2b1u) &  $\sigma^*_{\mathrm{s-s}}$  &  -12.067      &
	    &  $\sigma^*_{\mathrm{s-p}}$  &  -13.927  \\
      HOMO-5 (2ag)  &  $\sigma_{\mathrm{s-s}}$    &  -17.747      &
	    &  $\sigma^*_{\mathrm{s-p}}$  &  -13.927  
    \end{tabular}
\end{table}

% Include assignmen 5.1 results please
The HOMO-LUMO gap of the neutral singlet complex was +4.29684 eV, whereas the
neutral triplet complex yielded a negative gap.  A doublet state was the only
reasonable spin state for the monoanion complex without exciting the electron.
The doublet state had a smaller HOMO-LUMO gap than the neutral closed-shell
complex of +0.74691 eV, and the excited quartet state also had a negative gap.
The closed shell spin state of the neutral complex makes it EPR inactive,
whereas the lone electron on the monoanion makes it EPR active.  The bond length
between the metal centers in the complex were determined to be 354.55pm and
346.19pm for the neutral and monoanion complex, respectively.  This corresponds
to a 0.724\% error and a 0.344\% error respectively when compared to Dumas et
al. \cite{dumas}

\begin{figure}[h]
	\centering
	\includegraphics[width=\textwidth]{../images/lanthanide/merged.jpg}
\end{figure}

% Suggest minimal molecular orbital later.

\section{Conclusions} % Fourth

\begin{thebibliography}{80}
	\bibitem{dumas} Megan T. Dumas et al. “Synthesis and reductive
		chemistry of bimetallic and trimetallic rare- 2 earth
		metallocene hydrides with (C5H4SiMe3)1- ligands”. In: J.
		Organomet. Chem. 849-850 3 (2017), pp.  38–47. doi:
		10.1016/j.jorganchem.2017.05.057.
\end{thebibliography}

\section*{Appendix}
\begin{tabular}{cc}
	Structure & HOMO-LUMO Gap (eV) \\
	Anion & $+0.74691$ \\
	Neutral Singlet & $+4.29684$ \\
	Neutral Triplet & $-2.38537$ \\
	Neutral Closed & $+4.29805$
\end{tabular}

\begin{tabular}{cc}
	Structure & Y-Y Distance (pm) \\
	Neutral & $354.55$ \\
	Neutral Exp & $352$ \\
	Anion & $347.19$ \\
	Anion Exp & $346$
\end{tabular}

\end{document}

