\documentclass{article}
\usepackage{amsmath}

\title{Homework 9}
\author{Physics 112A}
\date{}

\begin{document}

\maketitle

\textbf{Problem 5.37}
A circular loop of wire, with radius $R$, lies in the $xy$ plane (centered at the origin) and carries a current $I$ running counterclockwise as viewed from the positive $z$-axis.

\textbf{(a)}
What is its magnetic dipole moment ?

\begin{equation*}
\begin{split}
	m & = I \int da^\prime \\
	& = \boxed{I \pi R^2 \hat{z}}
\end{split}
\end{equation*}

\textbf{(b)}
What is the (approximate) magnetic field at point far from the origin ?

\begin{equation*}
\begin{split}
	A & = \frac{\mu_0}{4 \pi} \frac{m \hat{z} \times \hat{r}}{r^2} \\
	& = \frac{\mu_0}{4 \pi} \frac{m [cos\theta \hat{r} - sin\theta \hat{\theta}] \times \hat{r}}{r^2} \\
	& = \frac{\mu_0}{4 \pi} \frac{m sin\theta}{r^2} \hat{\phi} \\
	B & = \frac{\mu_0 m}{4 \pi} (\frac{1}{r sin\theta} \frac{\partial}{\partial \theta} [sin\theta \frac{sin\theta}{r^2}] \hat{r} - \frac{1}{r} \frac{\partial}{\partial r} [r \frac{sin\theta}{r^2}] \hat{\theta}) \\
	& = \frac{\mu_0 m}{4 \pi r^3} [2 cos\theta \hat{r} + sin\theta \hat{\theta}] \\
	& = \boxed{\frac{\mu_0 I R^2}{4 r^3} [2 cos\theta \hat{r} + sin\theta \hat{\theta}]}
\end{split}
\end{equation*}

\textbf{(c)}
Show that, for points on the $z$-axis, your answer is consistent with the exact field (Ex. 5.6), when $z >> R$.

At the $z$-axis, $\hat{r} = \hat{z}$ and $\theta = 0$, so

\begin{equation*}
\begin{split}
	B & = \frac{\mu_0 I R^2}{2 z^3} \hat{z}
\end{split}
\end{equation*}

From Ex. 5.6:

\begin{equation*}
\begin{split}
	B(z) & = \frac{\mu_0 I}{2} \frac{R^2}{(R^2 + z^2)^{\frac{3}{2}}} \hat{z} \\
	& = \frac{\mu_0 I R^2}{2 z^3} \hat{z}
\end{split}
\end{equation*}

\textbf{Problem 5.39}

\textbf{(a)}
A phonograph record of radius $R$, carrying uniform surface charge $\sigma$, is rotating at constant angular velocity $\omega$.
Find its magnetic dipole moment.

\begin{equation*}
\begin{split}
	I & = \int \sigma v dr \\
	& = \sigma \omega \int_0^R r dr \\
	& = \frac{1}{2} \sigma \omega R^2 \\
	m & = \frac{1}{2} \sigma \omega R^2 \int_0^R \pi r dr \\
	& = \boxed{\frac{1}{4} \pi \sigma \omega R^4 \hat{z}}
\end{split}
\end{equation*}

\textbf{(b)}
Find the magnetic dipole moment of the spinning spherical shell in Ex. 5.11.
Show that for points $r > R$ the potential is that of a perfect dipole.

\begin{equation*}
\begin{split}
	dI & = \frac{dq}{t} \\
	& = \frac{\sigma \omega 2 \pi R^2 sin\theta d\theta}{2 \pi} \\
	& = \omega \sigma R^2 sin\theta d\theta \\
	a & = \pi (R sin\theta)^2 \\
	m & = \pi \sigma \omega R^4 \int_0^\pi sin^3\theta d\theta \\
	& = \pi \sigma \omega R^4 [\frac{1}{3} cos\theta - cos\theta]_0^\pi \\
	& = \boxed{\frac{4}{3} \pi \sigma \omega R^4 \hat{z}}
\end{split}
\end{equation*}

\begin{equation*}
\begin{split}
	A & = \frac{\mu_0}{4 \pi} \frac{4}{3} \pi \sigma \omega R^4 \hat{\phi} \\
	& = \frac{1}{3 r^2} \mu_0 \sigma \omega R^4 sin\theta \hat{\phi}
\end{split}
\end{equation*}

\textbf{5.44}
A current $I$ flows to the right through a rectangular bar of conducting material, in the prescence of a uniform magnetic field $B$ pointing out of the page.

\textbf{(a)}
The moving charges will be deflected $\boxed{\text{downwards.}}$

\textbf{(b)}
Find the resulting potential difference (the Hall voltage) between the top and bottom of the bar, in terms of $B$, $v$ (the speed of the charges), and the relevant dimensions of the bar.

\begin{equation*}
\begin{split}
	q(E + v B) & = 0 \\
	E & = - v B \\
	V & = - \int_0^t E \cdot dl \\
	& = \boxed{v B t}
\end{split}
\end{equation*}

\textbf{(c)}
The potential would be greater at the top plate instead.

\end{document}
