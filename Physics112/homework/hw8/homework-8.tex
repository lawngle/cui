\documentclass{article}

\usepackage{amsmath}

\title{Homework 8}
\author{Physics 112A}
\date{}

\begin{document}

\maketitle

\textbf{Problem 5.12}
Use the result of Ex. 5.6 to calculate the magnetic field at the center of a uniformly charged spherical shell, of radius $R$ and total charge $Q$, spinning at constant angular velocity $\omega$
$$B(z) = \frac{\mu_0 I}{4 \pi} \frac{cos\theta}{r^2} 2 \pi R = \frac{\mu_0 I}{2} \frac{R^2}{(R^2 + z^2)^{\frac{3}{2}}}$$
% width = r d\theta

$\theta$ is from the center of the sphere instead of from the ring, so $cos\theta \rightarrow sin\theta$.
\begin{equation*}
\begin{split}
	R & \rightarrow R sin\theta \\
	dI & = K R d\theta \\
	& = \sigma v R d\theta \\
	& = \frac{Q}{4 \pi R^2} R sin\theta \omega R d\theta \\
	& = \frac{Q \omega}{4 \pi} sin\theta d\theta \\
	dB & = \frac{2 \pi \mu_0}{4 \pi} \frac{R sin^2\theta}{R^2} dI \\
	& = \frac{\mu_0}{2 R} sin^2\theta \frac{Q \omega}{4 \pi} sin\theta d\theta \\
	& = \frac{Q \omega \mu_0}{8 \pi R} \int_0^\pi sin^3\theta d\theta \\
	& = \frac{Q \omega \mu_0}{8 \pi R} [\frac{1}{3} cos^3\theta - cos\theta]_0^\pi] \\
	& = \boxed{\frac{Q \omega \mu_0}{6 \pi R}}
\end{split}
\end{equation*}

\textbf{Problem 5.13}
Suppose you have two infinite straight-line charges $\lambda$, a distance $d$ apart, moving along at a constant speed $v$.
How great would $v$ have to be in order for the magnetic attraction to balance the electrical repulsion ?
Work out the actual number.
Is this a reasonable sort of speed ?

\begin{equation*}
\begin{split}
	E & = \frac{\lambda L}{2 \pi \epsilon_0 d^2} \\
	B & = \frac{\mu_0 I L}{4 \pi d^2} \\
	F_C & = - F_L \\
	\frac{1}{4 \pi \epsilon_0} \frac{(\lambda L)^2}{d^2} & = - \lambda L (\frac{\lambda L}{2 \pi \epsilon_0 d^2} + v \frac{\mu_0 I L}{4 \pi d^2}) \\
	\frac{1}{4 \epsilon_0} & = - \frac{1}{2 \epsilon_0} - \frac{\mu_0 v^2}{4} \\
	1 & = - 2 - \mu_0 \epsilon_0 v^2 \\
	 v^2 & = \frac{1}{\mu_0 \epsilon_0} \\
	 v & = \boxed{\frac{1}{\sqrt{\mu_0 \epsilon_0}}}
\end{split}
\end{equation*}

The speed of light is also $\frac{1}{\sqrt{\mu_0 \epsilon_0}}$, so this is not possible. 

\textbf{Problem 5.14}
A steady current $I$ flows down a long cylindrical wire of radius $a$.
Find the magnetic field, both inside and outside the wire if:

\textbf{(a)}
the current is uniformly distributed over the surface of the wire.

When $s < a$:
\begin{equation*}
\begin{split}
	\int B \cdot dl & = \mu_0 I \\
	B 2 \pi s & = 0 \\
	B & = \boxed{0}
\end{split}
\end{equation*}

When $s > a$:
\begin{equation*}
\begin{split}
	\int B \cdot dl & = \mu_0 I \\
	B 2 \pi s & = \mu_0 I \\
	B & = \boxed{\frac{\mu_0 I}{2 \pi s} \hat{\phi}}
\end{split}
\end{equation*}

\textbf{(b)}
the current is distributed in such a way that $J$ is proportional to $s$, the distance from the axis.

\begin{equation*}
\begin{split}
	J(s) & = k s \\
	I & = \int_0^a k s \pi s ds \\
	& = \frac{1}{3} k \pi [s^3]_0^a \\
	& = \frac{1}{3} k \pi a^3 \\
	k & = \frac{3 I}{a^3 \pi}
\end{split}
\end{equation*}

When $s < a$:
\begin{equation*}
\begin{split}
	I_{\text{enc}} & = \int_0^s \frac{3 I}{a^3 \pi} s \pi s ds \\
	& = \frac{3 I}{a^3} \frac{1}{3} [s^3]_0^s \\
	& = I \frac{s^3}{a^3} \\
	\int B \cdot dl & = \mu_0 I \\
	B 2 \pi s & = \mu_0 I \frac{s^3}{a^3} \\
	B & = \boxed{\frac{\mu_0 I}{2 \pi} \frac{s^2}{a^3} \hat{\phi}}
\end{split}
\end{equation*}

When $s > a$:
\begin{equation*}
\begin{split}
	\int B \cdot dl & = \mu_0 I_{\text{enc}} \\
	B 2 \pi s & = \mu_0 I \\
	B & = \boxed{\frac{\mu_0 I}{2 \pi s} \hat{\phi}}
\end{split}
\end{equation*}

\textbf{Problem 5.25}
Find the magnetic vector potential of a finite segment of straight wire carrying a current $I$.
[Put the wire on the $z$-axis, from $z_1$ to $z_2$, and use Eq. 5.66.]
Check that your answer is consistent with Eq. 5.37.

Eq. 5.66:
$$A = \frac{\mu_0 I}{4 \pi} \int \frac{dl}{r}$$

Using cylindrical coordinates:
\begin{equation*}
\begin{split}
	r^2 & = z^2 + s^2 \\
	A & = \frac{\mu_0 I}{4 \pi} \int_{z_1}^{z_2} \frac{dz}{\sqrt{z^2 + s^2}} \hat{z} \\
	& = \frac{\mu_0 I}{4 \pi} \int_{z_1}^{z_2} [ln(z + \sqrt{s^2 + z^2})]_{z_1}^{z_2} \\
	& = \boxed{\frac{\mu_0 I}{4 \pi} ln[\frac{z_2 + \sqrt{s^2 + z_2^2}}{z_1 + \sqrt{s^2 + z_1^2}}] \hat{z}}
\end{split}
\end{equation*}

Checking consistency:
\begin{equation*}
\begin{split}
	B & = \nabla \times A \\
	& = - \frac{\partial}{\partial s} A_z \hat{\phi} \\
	& = - \frac{\mu_0 I}{4 \pi} \frac{\partial}{\partial s} [ln[\frac{z_2 + \sqrt{s^2 + z_2^2}}{z_1 + \sqrt{s^2 + z_1^2}}]] \\
	& = - \frac{\mu_0 I}{4 \pi} [\frac{s}{z_2 + \sqrt{s^2 + z_2^2}} \frac{1}{\sqrt{s^2 + z_2^2}} - \frac{s}{z_2 + \sqrt{s^2 + z_1^2}} \frac{1}{\sqrt{s^2 + z_1^2}}] \\
	& = - \frac{\mu_0 I s}{4 \pi} [\frac{z_2 - \sqrt{s^2 + z_2^2}}{z_2^2 - (s^2 + z_2^2)} \frac{1}{\sqrt{s^2 + z_2^2}} - \frac{z_1 - \sqrt{s^2 + z_1^2}}{z_1^2 - (s^2 + z_1^2)} \frac{1}{\sqrt{s^2 + z_1^2}}] \\
	& = - \frac{\mu_0 I}{4 \pi s} [\frac{z_2 - \sqrt{s^2 + z_2^2}}{\sqrt{s^2 + z_2^2}} - \frac{z_1 - \sqrt{s^2 + z_1^2}}{\sqrt{s^2 + z_1^2}}] \\
	& = - \frac{\mu_0 I}{4 \pi s} [\frac{z_2}{\sqrt{s^2 + z_2^2}} - \frac{z_1}{\sqrt{s^2 + z_1^2}}] \\
	& = \boxed{- \frac{\mu_0 I}{4 \pi s} [sin\theta_2 - sin\theta_1] \hat{\phi}}
\end{split}
\end{equation*}

\textbf{Problem 5.26}

\textbf{(a)}
What current density would produce the vector potential $A = k \hat{\phi}$ (where $k$ is a constant), in cylindrical coordinates ?

\begin{equation*}
\begin{split}
	B & = \nabla \times A \\
	& = \frac{1}{s} \frac{\partial}{\partial s} [s A_{\phi}] \hat{z} \\
	& = \frac{k}{s} \hat{z} \\
	J & = \frac{1}{\mu_0} \nabla \times B \\
	& = - \frac{1}{\mu_0} \frac{\partial}{\partial s} B_z \hat{\phi} \\
	& = \boxed{\frac{k}{\mu_0 s^2} \hat{\phi}}
\end{split}
\end{equation*}

\textbf{(b)}
Consider an azimuthally symmetric magnetic field; it points in the $z$ direction, and its magnitude is a function only of $s$.
Check that
$$A = A(s) \hat{\phi} \text{ where } A(s) = \frac{1}{s} \int_0^s B(s^\prime) s^\prime ds^\prime$$
by calculating its divergence and curl.
(This generalizes Ex 5.12.)

\begin{equation*}
\begin{split}
	\nabla \cdot A & = \nabla \cdot \frac{1}{s} \int_0^s B(s^\prime) s^\prime ds^\prime \hat{\phi} \\
	& = \frac{1}{s} \frac{\partial}{\partial \phi} [\frac{1}{s} \int_0^s B(s^\prime) s^\prime ds^\prime] \\
	& = \boxed{0} \\
	\nabla \times A & = \nabla \times \frac{1}{s} \int_0^s B(s^\prime) s^\prime ds^\prime \hat{\phi} \\
	& = \frac{1}{s} \frac{\partial}{\partial s} [s \frac{1}{s} \int_0^s B(s^\prime) s^\prime ds^\prime] \hat{z} \\
	& = \frac{1}{s} B(s) s \\
	& = \boxed{B(s)}
\end{split}
\end{equation*}

\textbf{Problem 5.27}
If $B$ is uniform, show that $A(r) = - \frac{1}{2} (r \times B)$ works.
That is, check that $\nabla \cdot A = 0$ and $\nabla \times A = B$.
Is this result uique, or are there other functions with the same divergence and curl ?

\begin{equation*}
\begin{split}
	\nabla \cdot A & = - \frac{1}{2} \nabla \times (r \times B) \\
	& = - \frac{1}{2} [B \cdot (\nabla \times r) - r \cdot (\nabla \times B)] \\
	& = \boxed{0} \\
	\nabla \times A & = - \frac{1}{2} \nabla \times (r \times B) \\
	& = - \frac{1}{2} [(B \cdot \nabla) r - (r \cdot \nabla) B + r (\nabla \cdot B) - B (\nabla \cdot r)] \\
	& = - \frac{1}{2} [(B_x \frac{\partial r_x}{\partial x} + B_y \frac{\partial r_y}{\partial y} + B_z \frac{\partial r_z}{\partial z}) - B (3)] \\
	& = - \frac{1}{2} [B - 3 B] \\
	& = \boxed{B}
\end{split}
\end{equation*}

From Problem 5.26.b:
\begin{equation*}
\begin{split}
	A & = \frac{B}{s} \int_0^s s^\prime ds^\prime \hat{\phi} \\
	& = \boxed{\frac{1}{2} B s \hat{\phi}} \\
	\nabla \cdot A & = \frac{1}{s} \frac{\partial}{\partial \phi} \frac{1}{2} B s \\
	& = 0 \\
	\nabla \times A & = \frac{1}{s} \frac{\partial}{\partial s} [s \frac{1}{2} B s] \\
	& = B
\end{split}
\end{equation*}

\textbf{Problem 5.28}

\textbf{(a)}
By whatever means you can think of (short of looking it up), find the vector potential a distance $s$ from an infinite straight wire carrying a current $I$.
Check that $\nabla \cdot A = 0$ and $\nabla \times A = B$.

\begin{equation*}
\begin{split}
	\int B \cdot dl & = \mu_0 I \\
	B & = \frac{\mu_0 I}{2 \pi s} \hat{\phi} \\
	\nabla \times A(s) \hat{z} & = \frac{\mu_0 I}{2 \pi s} \hat{\phi} \\
	- \frac{\partial}{\partial s} [A(s)] & = \frac{\mu_0 I}{2 \pi s} \hat{\phi} \\
	A(s) & = - \int_a^s \frac{\mu_0 I}{2 \pi s} ds \\
	& = \boxed{- \frac{\mu_0 I}{2 \pi} ln[\frac{s}{a}] \hat{z}} \\
	\nabla \cdot A & = \frac{\partial}{\partial z} [- \frac{\mu_0 I}{2 \pi} ln(\frac{s}{a})] \\
	& = 0 \\
	\nabla \times A & = - \frac{\partial}{\partial s} [- \frac{\mu_0 I}{2 \pi} ln(\frac{s}{a})] \\
	& = \frac{\mu_0 I}{2 \pi s} \hat{\phi} \\
	& = B
\end{split}
\end{equation*}

\textbf{(b)}
Find the magnetic potential inside the wire, if it has radius $R$ and the current is uniformly distributed.

\begin{equation*}
\begin{split}
	\int B \cdot dl & = \mu_0 I \\
	B 2 \pi s & = \mu_0 J \pi s^2 \\
	B & = \frac{1}{2} \mu_0 J s \hat{\phi} \\
	- \frac{\partial}{\partial s} [A(s)] & = \frac{1}{2} \mu_0 J s \hat{\phi} \\
	A(s) & = - \frac{1}{2} \mu_0 \frac{I}{\pi R^2} \int_a^s s ds \\
	& = \boxed{- \frac{\mu_0 I}{4 \pi R^2} (s^2 - a^2) \hat{z}}
\end{split}
\end{equation*}

\end{document}
