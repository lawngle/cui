\documentclass{article}
\usepackage{amsmath}

\title{Homework 10}
\author{Physics 112A}
\date{}

\begin{document}

\maketitle

\textbf{Problem 6.8}
A very long circular cylinder of radius $R$ carries a magnetization $M = k s^2 \hat{\phi}$, where $k$ is a constant.
Find the magnetic field due to $M$, for points inside and outside the cylinder (and far from the ends).

\begin{equation*}
\begin{split}
	J_b & = \nabla \times M \hat{\phi} \\
	& = \frac{1}{s} \frac{\partial}{\partial s} [s k s^2] \hat{z} \\
	& = 3 k s \hat{z} \\
	K_b & = M \hat{\phi} \times \hat{n} \\
	& = - k s^2 \hat{z} \\
	& = - k R^2 \hat{z}
\end{split}
\end{equation*}

When $s < R$:
\begin{equation*}
\begin{split}
	\int B \cdot dl & = \mu_0 \int_0^s J_b da \\
	B 2 \pi s & = \mu_0 \int_0^s 3 k s 2 \pi s ds \\
	B & = \frac{3 \mu_0 k}{s} [\frac{1}{3} s^3]_0^s \\
	& = \boxed{\mu_0 k s^2 \hat{\phi}}
\end{split}
\end{equation*}

When $s > R$:
\begin{equation*}
\begin{split}
	\int B \cdot dl & = \mu_0 (\int_0^R J_b da + \int_0^R K_b dl) \\
	B 2 \pi s & = \mu_0 (\int_0^R 3ks 2 \pi s ds + \int_0^R - k R^2 2 \pi ds) \\
	& = \mu_0 (2 \pi k R^3 - 2 \pi k R^2 [s]_0^R) \\
	B & = \boxed{0}
\end{split}
\end{equation*}

\textbf{Problem 6.10}
An iron rod of length $L$ and square cross section (side length $a$) is given a uniform longitudinal magnetization $M$, and then bent around into a circle with a narrow gap (width $w$).
Find the magnetic field at the center of the gap, assuming $w << a << L$.
[Hint: Treat it as the superposition of a complete torus plus a square loop with revered current.]

\textbf{RETURN LATER}

\textbf{Problem 6.12}
An infinitely long cylinder, of radius $R$, carries a "frozen-in" magnetization, parallel to the axis,
$$M = ks \hat{z}$$
where $k$ is a constant and $s$ is the distance from the axis; there is no free current anywhere.
Find the magnetic field inside and outside the cylinder by two different methods:

\textbf{(a)}
Locate all the bound currents, and calculate the field they produce.

\begin{equation*}
\begin{split}
	J_b & = \nabla \times M \hat{z} \\
	& = - \frac{\partial}{\partial s} k s \hat{\phi} \\
	& = - k \hat{\phi} \\
	K_b & = M \hat{z} \times \hat{n} \\
	& = k s \hat{\phi} \\
	& = k R \hat{\phi}
\end{split}
\end{equation*}

When $s < R$:
\begin{equation*}
\begin{split}
	\int B \cdot dl & = \mu_0 (\int J_b da + \int K_b dl) \\
	B l & = \mu_0 (- k l [R - s] + k l R) \\
	B & = \boxed{\mu_0 k s \hat{z}}
\end{split}
\end{equation*}

When $s > R$:
\begin{equation*}
\begin{split}
	\int B \cdot dl & = \mu_0 (\int J_b da + \int K_b dl) \\
	B & = \boxed{0}
\end{split}
\end{equation*}

\textbf{(b)}
Use Ampere's law to find $H$, and then get $B$.

$H = 0$ since there is no free current anywhere.

When $s < R$:
\begin{equation*}
\begin{split}
	B & = \mu_0 M \\
	& = \boxed{\mu_0 k s \hat{z}}
\end{split}
\end{equation*}

When $s < R$:
\begin{equation*}
\begin{split}
	M & = 0 \\
	B & = \boxed{0}
\end{split}
\end{equation*}

\textbf{Problem 6.16}
A coaxial cable consists of two very long cylindrical tubes, separated by linear insulating material of magnetic susceptibility $\chi_m$.
A current $I$ flows down the inner conductor and returns along the outer one; in each case, the current distributes itself uniformly over the surface.
Find the magnetic field in the region between the tubes.
As a check, calculate the magnetization and the bound currents, and confirm that (together, of course, with the free currents) they generate the correct field.

\begin{equation*}
\begin{split}
	\int H \cdot dl & = I \\
	H & = \frac{I}{2 \pi s} \\
	B & = \mu H \\
	& = \boxed{\mu_0 (1 + \chi_m) \frac{I}{2 \pi s}}
\end{split}
\end{equation*}

Check:
\begin{equation*}
\begin{split}
	M & = \chi_m H \\
	& = \chi_m I \frac{1}{2 \pi s} \hat{\phi} \\
	J_b & = \nabla \times M \hat{\phi} \\
	& = \frac{1}{s} \frac{\partial}{\partial s} [s \chi_m I \frac{1}{2 \pi s}] \hat{z} \\
	& = 0 \\
	K_b & = M \hat{\phi} \times \hat{n} \\
	& = \chi_m I \frac{1}{2 \pi s} \hat{z} \\
	& = \chi_m I \frac{1}{2 \pi a} \hat{z} \\
	\int B \cdot dl & = \mu_0 (I + \int K_b dl) \\
	B 2 \pi s & = \mu_0 (I + \chi_m I \frac{1}{2 \pi a} 2 \pi a) \\
	B & = \mu_0 I (1 + \chi_m) \frac{1}{2 \pi s}
\end{split}
\end{equation*}

\end{document}
