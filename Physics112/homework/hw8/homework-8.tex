\documentclass{article}

\usepackage{amsmath}

\title{Homework 8}
\author{Physics 112A}
\date{}

\begin{document}

\maketitle

\textbf{Problem 5.12}
Use the result of Ex. 5.6 to calculate the magnetic field at the center of a uniformly charged spherical shell, of radius $R$ and total charge $Q$, spinning at constant angular velocity $\omega$
$$B(z) = \frac{\mu_0 I}{4 \pi} \frac{cos\theta}{r^2} 2 \pi R = \frac{\mu_0 I}{2} \frac{R^2}{(R^2 + z^2)^{\frac{3}{2}}}$$
% width = r d\theta

$\theta$ is from the center of the sphere instead of from the ring, so $cos\theta \rightarrow sin\theta$.
\begin{equation*}
\begin{split}
	R & \rightarrow R sin\theta \\
	dI & = K R d\theta \\
	& = \sigma v R d\theta \\
	& = \frac{Q}{4 \pi R^2} R sin\theta \omega R d\theta \\
	& = \frac{Q \omega}{4 \pi} sin\theta d\theta \\
	dB & = \frac{2 \pi \mu_0}{4 \pi} \frac{R sin^2\theta}{R^2} dI \\
	& = \frac{\mu_0}{2 R} sin^2\theta \frac{Q \omega}{4 \pi} sin\theta d\theta \\
	& = \frac{Q \omega \mu_0}{8 \pi R} \int_0^\pi sin^3\theta d\theta \\
	& = \frac{Q \omega \mu_0}{8 \pi R} [\frac{1}{3} cos^3\theta - cos\theta]_0^\pi \\
	& = \boxed{\frac{Q \omega \mu_0}{6 \pi R}}
\end{split}
\end{equation*}

\textbf{Problem 5.13}
Suppose you have two infinite straight-line charges $\lambda$, a distance $d$ apart, moving along at a constant speed $v$.
How great would $v$ have to be in order for the magnetic attraction to balance the electrical repulsion ?
Work out the actual number.
Is this a reasonable sort of speed ?

\begin{equation*}
\begin{split}
	E & = \frac{\lambda L}{2 \pi \epsilon_0 d^2} \\
	B & = \frac{\mu_0 I L}{4 \pi d^2} \\
	F_C & = - F_L \\
	\frac{1}{4 \pi \epsilon_0} \frac{(\lambda L)^2}{d^2} & = - \lambda L (\frac{\lambda L}{2 \pi \epsilon_0 d^2} + v \frac{\mu_0 I L}{4 \pi d^2}) \\
	\frac{1}{4 \epsilon_0} & = - \frac{1}{2 \epsilon_0} - \frac{\mu_0 v^2}{4} \\
	1 & = - 2 - \mu_0 \epsilon_0 v^2 \\
	 v^2 & = \frac{1}{\mu_0 \epsilon_0} \\
	 v & = \boxed{\frac{1}{\sqrt{\mu_0 \epsilon_0}}}
\end{split}
\end{equation*}

The speed of light is also $\frac{1}{\sqrt{\mu_0 \epsilon_0}}$, so this is not possible. 

\textbf{Problem 5.14}
A steady current $I$ flows down a long cylindrical wire of radius $a$.
Find the magnetic field, both inside and outside the wire if:

\textbf{(a)}
the current is uniformly distributed over the surface of the wire.

When $s < a$:
\begin{equation*}
\begin{split}
	\int B \cdot dl & = \mu_0 I \\
	B 2 \pi s & = 0 \\
	B & = \boxed{0}
\end{split}
\end{equation*}

When $s > a$:
\begin{equation*}
\begin{split}
	\int B \cdot dl & = \mu_0 I \\
	B 2 \pi s & = \mu_0 I \\
	B & = \boxed{\frac{\mu_0 I}{2 \pi s} \hat{\phi}}
\end{split}
\end{equation*}

\textbf{(b)}
the current is distributed in such a way that $J$ is proportional to $s$, the distance from the axis.

\begin{equation*}
\begin{split}
	J(s) & = k s \\
	I & = \int_0^a k s \pi s ds \\
	& = \frac{1}{3} k \pi [s^3]_0^a \\
	& = \frac{1}{3} k \pi a^3 \\
	k & = \frac{3 I}{a^3 \pi}
\end{split}
\end{equation*}

When $s < a$:
\begin{equation*}
\begin{split}
	I_{\text{enc}} & = \int_0^s \frac{3 I}{a^3 \pi} s \pi s ds \\
	& = \frac{3 I}{a^3} \frac{1}{3} [s^3]_0^s \\
	& = I \frac{s^3}{a^3} \\
	\int B \cdot dl & = \mu_0 I \\
	B 2 \pi s & = \mu_0 I \frac{s^3}{a^3} \\
	B & = \boxed{\frac{\mu_0 I}{2 \pi} \frac{s^2}{a^3} \hat{\phi}}
\end{split}
\end{equation*}

When $s > a$:
\begin{equation*}
\begin{split}
	\int B \cdot dl & = \mu_0 I_{\text{enc}} \\
	B 2 \pi s & = \mu_0 I \\
	B & = \boxed{\frac{\mu_0 I}{2 \pi s} \hat{\phi}}
\end{split}
\end{equation*}

\textbf{Problem 5.25}
Find the magnetic vector potential of a finite segment of straight wire carrying a current $I$.
[Put the wire on the $z$-axis, from $z_1$ to $z_2$, and use Eq. 5.66.]
Check that your answer is consistent with Eq. 5.37.

Eq. 5.66:
$$A = \frac{\mu_0 I}{4 \pi} \int \frac{dl}{r}$$

Using cylindrical coordinates:
\begin{equation*}
\begin{split}
	r^2 & = z^2 + s^2 \\
	A & = \frac{\mu_0 I}{4 \pi} \int_{z_1}^{z_2} \frac{dz}{\sqrt{z^2 + s^2}} \hat{z} \\
	& = \frac{\mu_0 I}{4 \pi} \int_{z_1}^{z_2} [ln(z + \sqrt{s^2 + z^2})]_{z_1}^{z_2} \\
	& = \boxed{\frac{\mu_0 I}{4 \pi} ln[\frac{z_2 + \sqrt{s^2 + z_2^2}}{z_1 + \sqrt{s^2 + z_1^2}}] \hat{z}}
\end{split}
\end{equation*}

Checking consistency:


\end{document}
